\section{Comportamento}
Per comporamento del sito è gestito lato server dai file PHP e lato client da JavaScript.

\subsection{Javascript}
JS......

\subsubsection{Funzionalità}
......la lista delle funzioni:
\begin{itemize}
	\item \textbf{nomeFUNZIONE():} descrizione.....\\
\end{itemize}

\subsection{PHP}
Il backend del server è stato sviluppato con script PHP e database relazionale MariaDB;

\subsubsection{Classi}
\begin{itemize}
	\item DBAccess: è la classe che gestisce tutte le interazioni col database.
	\item User
	\item Game
	\item News
	\item GameNews
	\item Comment
	\item Image
\end{itemize}

\paragraph{DBAccess} 
La classe DBAccess è abbastanza complessa da richiedere un approfondimento.
Essa offre per metodi per la lettura e per la scrittura dei dati rappresentati dalle altre 6 classi.

I metodi più importanti sono:
\begin{itemize}
	\item \textbf{\texttt{nomeMetodo()}}: descrizione;

\end{itemize}


