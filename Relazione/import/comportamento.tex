\section{Comportamento}
Il comporamento del sito è gestito lato server da PHP e lato client da JavaScript.

\subsection{JavaScript}
Abbiamo usato un unico file JavaScript nominato checkForm.js, contenente principalmente funzioni che svolgono controlli sull'input delle form.
Di seguito riportiamo le funzioni con una breve spiegazione:

\begin{itemize}
	\item \textbf{validateForm(): } esegue un controllo lato client dell'input della form. Viene chiamata in tutte le pagine che contengono delle form per l'inserimento o la modifica dei dati lato admin e nelle pagine di login e registrazione lato utente.
	Per facilitare eventuali aggiunte di campi input nel sito, la funzione si appoggia all'array \texttt{dettagliForm} contenente come chiavi gli id dei campi input a cui vengono associati array contenenti le espressioni regolari per la verifica dell'input e i corrispondenti messaggi d'errore.
	La funzione procede chiamando \texttt{validateInput(input)} per ogni elemento di \texttt{dettagliForm} trovato nella pagina di invocazione, prosegue chiamendo \texttt{checkChecked()} ed infine ritorna un valore booleano in base all'esito dei controlli effettuati; \\

	\item \textbf{validateInput(input): } effettua il controllo del campo input. Nel caso sia già presente un messaggio (di errore o di conferma del corretto inserimento), lo rimuove. Gestisce in seguito i casi limite, non gestibili tramite l'array di supporto: campi input di tipo file o data, il campo \texttt{repeatedpassword} e i campi opzionali. Effettua quindi il controllo dei restanti campi input sfruttando le espressioni regolari presenti in \texttt{dettagliForm} e procede ritornando il valore ricevuto dall'invocazione di \texttt{showMessage(input,stato)}; \\

	\item \textbf{showMessage(input, stato): } crea un messaggio che comunica all'utente se l'input inserito è corretto o meno. Abbiamo deciso di fornire 2 meccanismi di feedback all'utente, presentando sia un messaggio testuale che una colorazione del bordo del campo input, in questo modo possiamo facilitare la compilazione delle form a tutte le classi di utenti. \\

	\item \textbf{checkPassword(): } controlla che gli input in \texttt{password} e in \texttt{repetedpassword} siano uguali e invoca opportunamente \texttt{showMessage(input,stato)}; \\

	\item \textbf{checkChecked(): } effettua un controllo dei campi checkbox e radio sfruttando l'array di supporto \texttt{spuntabili} in modo analogo a \texttt{validateForm()}. Crea un array contenente coppie chiave-valore nella forma: id del campo input, valore booleano (che indica se il rispettivo campo input è stato spuntato). Tale array viene poi fornito come parametro a \texttt{showMessageCheckbox(sezioni)}; \\

	\item \textbf{showMessageCheckbox(sezioni): } opera in modo analogo a \texttt{showMessage(input, stato)}, creando però dei messaggi che comunicano all'utente se i campi input (di tipo checkbox o radio) sono stati spuntati o meno; \\

	\item \textbf{handleClick(): } fa comparire/scomparire il campo\n'\texttt{Gioco trattato}" se viene selezionata/deselezionata la categoria\n\texttt{Giochi}' in \texttt{Tipologia} nella form per l'inserimento di una notizia; \\

	\item \textbf{removeNoJs(): }  viene invocata al completamento del caricamento della pagina, rimuove la classe\n'no-js" dal tag \texttt{<html>}. Tale classe è stata utilizzata per la definizione di alcune regole \texttt{css} che gestiscono la casistica in cui JavaScript è disattivato. Ad esempio nel caso del menù principale nella versione mobile, con JS disattivato, viene rimosso il pulsante e viene mostrata la tendina aperta; \\

	\item \textbf{checkNotEmpty(id): } controlla che il campo \texttt{input} con id =\n'id" non sia vuoto o composto da spazi; \\

	\item \textbf{responsiveMenu(): } gestisce l'apertura e chiusura del menù a tendina nella versione mobile.
	E' associata ad un evento \texttt{onclick}, scelto perchè gestisce anche l'evento \texttt{ontouch} per dispositivi mobile.
	Il pulsante è situato nel top-right della pagina, dove è più facilmente raggiungibile da un utente mobile; \\

	\item \textbf{preparaFiltri(): } viene chiamata al completamento del caricamento della pagina \texttt{Giochi}. Gestisce l'evento onclick dei button tramite event delegation nel \texttt{div} con classe\n'container". All'evento onclick di un bottone dei filtri di ricerca lascia aperta la tendina contenente i filtri, mentre ad un secondo evento onclick all'interno del\n'container" la richiude; \\

	\item \textbf{checkAnni(): } effettua un controllo sull'intevallo degli anni di uscita dei videogiochi selezionato nella pagina giochi.
	Inserisce un messaggio d'errore se l'intervallo è inamissibile (es: da 2020 a 2016) e blocca il submit. \\
\end{itemize}




\subsection{BackEnd}
Il backend del server è stato sviluppato con script PHP e database relazionale MariaDB.

\subsubsection{Php}

Gli script che costituiscono il backend si possono suddividere in tre macrocategorie in base alla loro funzione:
\begin{itemize}
	\item interazione con il database;
	\begin{itemize}
		\item dbConnection.php
	\end{itemize}
	\item creazione e output delle pagine html;
	\begin{itemize}
		\item edit_gioco.php
		\item edit_notizia.php
		\item edit_profilo.php
		\item form_gioco.php
		\item form_notizia.php
		\item giochi.php
		\item gioco_notizie.php
		\item gioco_recensione.php
		\item gioco_scheda.php
		\item home.php
		\item lista_utenti.php
		\item login.php
		\item notizia.php
		\item notizie.php
		\item profilo.php
		\item registrati.php
		\item replacer.php
	\end{itemize}
	\item modellazione dei dati.
	\begin{itemize}
		\item comment.php
		\item game.php
		\item image.php
		\item news.php
		\item user.php
	\end{itemize}
\end{itemize}

\paragraph{Classi}
\begin{itemize}
	\item DBAccess: è la classe che gestisce tutte le interazioni col database.
	\item User
	\item Game
	\item News
	\item GameNews
	\item Comment
	\item Image
\end{itemize}

\subparagraph{DBAccess} 
La classe DBAccess è abbastanza complessa da richiedere un approfondimento.
Essa offre per metodi per la lettura e per la scrittura dei dati rappresentati dalle altre 6 classi.

I metodi più importanti sono:
\begin{itemize}
	\item \textbf{\texttt{nomeMetodo()}}: descrizione;

\end{itemize}

\subsubsection{Database}


  \includegraphics[width=\linewidth]{./img/diagramma_er.png}
  


