\section{Fase di Progettazione}
Nella fase di progettazione ci siamo basati sul Responsive Web Design, tramite la definizione di Media Query e di
breakpoint per gestire la problematica della variabilità dell'interfaccia, ottimizzando la visualizzazione in base al dispositivo.
Abbiamo deciso di utilizzare HTML5, rispettando la sintassi XML, garantendo in questo modo il supporto al passato e predisponendo un'ottimizzazione verso il futuro.
In particolare abbiamo utilizzato dei tag input introdotti con HTML5, come ad esempio datalist, per suppotrare le fasi di ricerca degli utenti e di inserimento e modifica dei contenuti degli admin. 

\subsection{Obiettivi}
Nella fasi di progettazione e sviluppo del sito sono stati perseguiti i seguenti obiettivi:
\begin{itemize}
	\item  \textbf{Separazione tre struttura e presentazione:} la struttura del sito è separata dalla parte di presentazione;
	\item  \textbf{Flessibilità:} il sito deve adattarsi alle differenti dimensioni dello schermo, in particolare risulta fondamentale garantire una buona fruizione da mobile, visto il numero crescente di accessi tramite dispositivi mobile;
	\item  \textbf{Accessibilità:} il sito deve essere accessibile a tutte le categorie di utenti.
\end{itemize}

\subsection{Attori}
Sono stati individuati tre tipi di attori:
\begin{itemize}
	\item \textbf{Utente non registrato:} può visitare tutti i contenuti del sito, ma non può commentare le pagine. Può effettuare la registrazione attraverso un form apposito e diventare così un utente registrato;
	\item \textbf{Utente registrato:} ha il permesso di accedere alla propria area riservata, dove può modificare le informazioni relative al suo account, può inoltre commentare nelle sezioni apposite;
	\item \textbf{Admin:} è un utente registrato con dei permessi aggiuntivi. Può eliminare utenti e inserire, modificare ed eliminare giochi e notizie.
\end{itemize}

\subsubsection{Procedura di modifica e/o inserimento} \label{subsection:modificainserimento}

\subsection{Struttura del sito}

\subsubsection{Header}

\subsubsection{Breadcrumb}

\subsubsection{Contenuto}

\paragraph{Pagina Home} 

pagine del sito.....

\paragraph{Profilo utente}

\subsubsection{Footer}

\subsubsection{Database}


\subsection{Accessibilità}
??????????????Per mantenere un alto livello di accessibilità abbiamo seguito lo standard WCAG 2.0.?????????????????????????????????

\subsubsection{Separazione tra contenuto, presentazione e struttura}

\subsubsection{Navigazione}

\paragraph{Breadcrumb} 

\paragraph{Testo e aiuti nascosti} 

\paragraph{Attributi HTML} 
