\section{Fase di Realizzazione}
Riportiamo le parti realizzate dai componenti del gruppo:
\begin{itemize}
	\item \textbf{Barilla Gianmarco:}
	\begin{enumerate}
		\item HTML di alcune pagine;
		\item CSS di alcune parti del sito;
		\item CSS per mobile e stampa;
		\item Stesura della relazione.
	\end{enumerate}
	\item \textbf{Chiarello Federico:}
	\begin{enumerate}
		\item HTML di alcune pagine;
		\item CSS di alcune parti del sito;
		\item File JavaScript;
		\item Stesura della relazione.
	\end{enumerate}
	\item \textbf{Furioso Patrick:}
	\begin{enumerate}
		\item HTML di alcune pagine;
		\item CSS di alcune parti del sito;
		\item Alcune funzioni JavaScript;
		\item Stesura della relazione.
	\end{enumerate}
	\item \textbf{Piacere Ivan:}
	\begin{enumerate}
		\item HTML di alcune pagine;
		\item Creazione Database e procedure;
		\item Creazione classi e pagine PHP;
		\item Stesura della relazione.
	\end{enumerate}
\end{itemize}


Inizialmente abbiamo creato un template contenente l'header e il footer, in modo che ogni componente potesse sviluppare le pagine a lui assegnate separatamente.
Abbiamo lavorato fin dall'inizio sullo stesso file CSS, definendo inizialmente le regole comuni e aggiungendo in seguito le parti relative a pagine spacifiche.
In un secondo momento è iniziato lo sviluppo del Database, del PHP e di JavaScript.
Le pagine create inizialmente come file HTML con del contenuto provvisorio, per poter definire l'aspetto della pagina, sono state modificate inserendo dei placeholder,
utilizzate dagli script PHP per la creazione effettiva delle pagine.
Infine abbiamo effettuato le attività di verifica, cercando inoltre di risolvere eventuali conflitti ed aumentare la qualità del codice prodotto.
Nello sviluppo sono insorte delle difficoltà legate ai conflitti creati dallo sviluppo contemporaneo di vari file.
