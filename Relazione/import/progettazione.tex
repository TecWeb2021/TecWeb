\section{Fase di Progettazione}
...Perchè abbiamo preferito HTML5...

\subsection{Attori}
Sono stati individuati tre tipi di attori:
\begin{itemize}
	\item \textbf{Utente non registrato:} può visitare tutti i contenuti del sito, ma non può commentare le pagine. Può effettuare la registrazione attraverso un form apposito e diventando così un utente registrato;
	\item \textbf{Utente registrato:} ha il permesso di accedere alla propria area riservata, dove può modificare le informazioni relative al suo account, può inoltre commentare nelle sezioni apposite;
	\item \textbf{Admin:} è un utente registrato con dei permessi aggiuntivi. Può eliminare utenti e inserire, modificare ed eliminare giochi e notizie.
\end{itemize}

\subsubsection{Procedura di modifica e/o inserimento} \label{subsection:modificainserimento}

\subsection{Struttura del sito}

\subsubsection{Header}

\subsubsection{Breadcrumb}

\subsubsection{Contenuto}

\paragraph{Pagina Home} 

pagine del sito.....

\paragraph{Profilo utente}

\subsubsection{Footer}

\subsubsection{Database}


\subsection{Accessibilità}
??????????????Per mantenere un alto livello di accessibilità abbiamo seguito lo standard WCAG 2.0.?????????????????????????????????

\subsubsection{Separazione tra contenuto, presentazione e struttura}

\subsubsection{Navigazione}

\paragraph{Breadcrumb} 

\paragraph{Testo e aiuti nascosti} 

\paragraph{Attributi HTML} 
